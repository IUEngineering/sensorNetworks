% o Inhoudelijke verkenning, kennis benodigd voordat met het ontwerp gestart kan worden (o.a. normen en regelgeving), 
% vaak a.d.h.v. gerelateerd werk.
% o Relevante onderzoeksvragen worden hierin uitgewerkt.
% o Welke literatuur en/of theorieën zijn relevant en wat betekent dit voor het
% ontwerp.
% o Overzicht van bestaande oplossingen van het probleem en waarom voldoen
% deze in dit specifieke geval wel/niet

Te lezen in Hoofdstuk \ref{opdracht} is dat er dus een probleem is met comfortabelheid in een ruimte. In de probleemstelling is al 
kort verteld welke grootheden hier invloed op hebben. Dit waren luchtkwaliteit, licht en geluid. Dit zijn ze daarentegen niet allemaal. 
In het onderzoek die is uitgevoerd door Kees Sietsma \cite{productiviteit} is te zien dat hij is gaan kijken naar de 
productiviteit van werknemers binnen een bedrijf. Uit dit onderzoek is gebleken dat werknemers minder productief zijn als de temperatuur
boven de 20 graden is. Bij elke graad staat dat de productiviteit daalt met 4\%. Ook is een werknemer gemiddeld 6\% productiver in een rustige
en stille ruimte dan een werknemer die in een ruimte zit waar veel geluid te horen is en waar veel velle kleuren te zien zijn. 

Buiten dit onderzoek is het ook 