\documentclass[12pt, a4paper, twoside]{article}

\usepackage{hvaTemplate}
\usepackage{csquotes}
\usepackage[dutch]{babel}
\usepackage[backend=biber, style=ieee]{biblatex}
\usepackage{multirow}
\addbibresource{references.bib}

\setAuthor{Tycho Jöbsis (500845792), Jochem Leijenhorst (500855372), Illya Ustenko (500845492) and Naomi Visser (500850314)}
\extraInfo{Implementatie van een dynamisch draadloos sensor netwerk}
\setTitle{Sensornetwerken}

\graphicspath{ {img/} }

\numberwithin{equation}{section}

\begin{document}
    \makeTitlepage
    \onecolumn
    \tableofcontents
    \newpage

    \section{Inleiding}
    In dit onderzoeksraamwerk wordt er gekeken naar wat de opdracht inhoudt en wat het probleem is. 
    Hieruit is een doelstelling opgesteld. Vanuit deze doelstelling worden deelvragen bedacht om deze in een 
    later hoofdstuk kort te beantwoorden met onderzoek. Hieruit volgt dan dus ook een netwerk waarin sommige 
    grootheden gemeten worden om hier vanuit feedback te kunnen geven aan de gebruiker. 

    \section{Opdracht}
    \subsection{Probleem stelling}
    Wanneer iemand in een kamer zit wilt diegene zich zo comfortabel mogelijk voelen. Dit komt omdat een persoon productiever is wanneer  
    diegene zich beter voelt in de kamer waarin diegene zich bevindt \cite{productiviteit}. Dit kan daarentegen negatief beinvloed worden door verschillende 
    factoren. Dit kan lucht kwaliteit zijn, warmte, geluid, etc. Het is daarom belangrijk om te weten wanneer het niet
    meer comfortabel is in een kamer en waarom.

\subsection{Doelstelling}
    Het probleem is dus dat een kamer door verschillende negatieve invloeden niet meer comfortabel is. Nu is het belangrijk om te weten waarom
    en wanneer dit zo is. Dit om uiteindelijk een signaal te geven en een eventuele oplossing om weer comfortabel te voelen in de kamer.
    Het doel zal dan ook zijn om iets te bedenken van een systeem om dit te kunnen monitoren en feedback over terug te geven.

\subsection{Opdracht}
De opdracht vanuit de hogeschool van amsterdam is om een netwerk te ontwikkelen die met elkaar kan praten en een basisstation heeft die met de 
informatie van de sensoren een conclusie kan trekken. Dit zal daarom ook als basis gebruikt worden om dit netwerk te kunnen realiseren. In dit 
onderzoeksraamwerk zal daarom ook eerst de focus liggen op het maken van een goed netwerk. Er zal ook gebruik gemaakt gaan worden van dummie data.

\subsection{Deelvragen}
\begin{itemize}
    \item Welke grootheden hebben invloed op de comfortabelheid van een persoon in een kamer?
    \item Op welke manier geven we feedback terug naar de gebruiker?
    \item Welke grootheden zorgen het meest voor oncomfortabelheid?
    \item Welke combinaties aan grootheden zorgen voor meer oncomfortabelheid?
\end{itemize}


    \newpage
    \section{Grootheden}
    Vanuit verschillende onderzoeken is er gekeken naar invloeden van buiten om de productiviteit van een persoon te beinvloeden. \cite{productiviteit} 
Hieruit is de volgende lijst en dus ook grootheden gekomen:

\begin{table}[ht]
    \begin{tabular}{p{3.3cm}||p{3cm}|p{3cm}|p{3cm}|p{3cm}}
         & Eenheid & Meetbereik & Nauwkeurigheid & meetfrequentie\\
         \hline
        Temperatuur &  $^{\circ}$C  & 10 tot 29 $^{\circ}$C & 90 \% & elk half uur\\
        Luchtvochtigheid & \% & 0 tot 100\%& 80 \% & elk half uur\\
        Luchtkwaliteit & Groen, Oranje en rood led & zie kleur tabel*& 80 \% & elke 10 minuten\\
        Geluid in de ruimte & dB & 0 - 90dB & 95 \% & elke 5 seconde**\\
        Licht & Lumen & 0 - 1600Lm & 90\% & elke 10 seconde
    \end{tabular}
    \caption{Grootheden}
    \label{tab:1}
\end{table}


 \vspace{1cm}
Wanneer 1 van deze grootheden bij een bepaalde waarde komt dat het niet meer comfortabel is, zie tabel \ref{tab:3}, dan zal er eerst een waarschuwing naar voren komen. Deze waarschuwing kan als knipperend lichtje of een kleine piep zijn. Wanneer meerdere grootheden tegelijketijd in de oncomfortabele zone zitten dan zal er een heftigere waarschuwing zijn. 

\vspace{1.5cm}
* Kleur tabel, kleuren inspiratie vanuit het KNMI \cite{Gezonde} \\

\begin{table}[ht]
    \begin{tabular}{c|c|c}
       Kleur & Eenheid & Meetbereik \\
       \hline
       Groen  &  microgram per cubic meter & $ < $ 16\\
       Oranje  &  microgram per cubic meter & 16 $->$ 25\\
       Rood & microgram per cubic meter & $>$ 25
    \end{tabular}
    \caption{kleur tabel}
    \label{tab:2}
\end{table}

\vspace{0.2cm}

** Hier moet een filter bij komen omdat er in een split seconde veel geluid kan zijn voor een seconde en dit niet meteen de alarmbellen moet afslaan. Een voorbeeld zou zijn om het gemiddelde van elke minuut te pakken.

\begin{table}[ht]
    \begin{tabular}{p{3cm}||p{3cm}|p{3cm}}
       Grootheid  & ondergrens & bovengrens \\
       \hline
       Temperatuur  & $<$ 15& $>$ 22 \cite{Temperatuur}\\
       Luchtvochtigheid & $<$ 40\% & $>$60\% \cite{Luchtvochtigheid}\\
       Luchtkwaliteit & n.v.t. & Rood \cite{Gezonde}\\
       Geluid in de ruimte & n.v.t. & $>$ 60dB \cite{Geluidsoverlast}\\
       Licht & n.v.t & $>$ 1500Lm \cite{Lumen}\\
    \end{tabular}
    \caption{Grenzen per grootheid}
    \label{tab:3}
\end{table}

    \newpage    
    \printbibliography

    \appendix

\end{document}