\subsection{Probleem stelling}
    Wanneer iemand in een kamer zit wilt diegene zich zo comfortabel mogelijk voelen. Dit komt omdat een persoon productiever is wanneer  
    diegene zich beter voelt in de kamer waarin diegene zich bevindt \cite{productiviteit}. Dit kan daarentegen negatief beinvloed worden door verschillende 
    factoren. Dit kan lucht kwaliteit zijn, warmte, geluid, etc. Het is daarom belangrijk om te weten wanneer het niet
    meer comfortabel is in een kamer en waarom.

\subsection{Doelstelling}
    Het probleem is dus dat een kamer door verschillende negatieve invloeden niet meer comfortabel is. Nu is het belangrijk om te weten waarom
    en wanneer dit zo is. Dit om uiteindelijk een signaal te geven en een eventuele oplossing om weer comfortabel te voelen in de kamer.
    Het doel zal dan ook zijn om iets te bedenken van een systeem om dit te kunnen monitoren en feedback over terug te geven.

\subsection{Opdracht}
De opdracht vanuit de hogeschool van amsterdam is om een netwerk te ontwikkelen die met elkaar kan praten en een basisstation heeft die met de 
informatie van de sensoren een conclusie kan trekken. Dit zal daarom ook als basis gebruikt worden om dit netwerk te kunnen realiseren. In dit 
onderzoeksraamwerk zal daarom ook eerst de focus liggen op het maken van een goed netwerk. Er zal ook gebruik gemaakt gaan worden van dummie data.

\subsection{Deelvragen}
\begin{itemize}
    \item Welke grootheden hebben invloed op de comfortabelheid van een persoon in een kamer?
    \item Op welke manier geven we feedback terug naar de gebruiker?
    \item Welke grootheden zorgen het meest voor oncomfortabelheid?
    \item Welke combinaties aan grootheden zorgen voor meer oncomfortabelheid?
\end{itemize}
