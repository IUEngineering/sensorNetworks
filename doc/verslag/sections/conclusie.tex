\section{Conclusie}
Het netwerk is nu gemaakt. Als eerste is het belangrijk om te kijken of de specificaties die zijn gemaakt ook zijn volbracht.
Het eerste punt dat aanwezig moest zijn in het netwerk is betrouwbaarheid. Dit is lastig om te meten, maar wel is er tijdens het ontwerpen 
gekeken naar hoevaak het programma gerest moest worden om een keer te werken. Dit is niet voorgekomen. Hierdoor kan er gezegd worden dat 
het hoogswaarschijnlijk betrouwbaar is. De echte test zal zijn tijdens de demo die gegeven gaat worden. Als het dan in één keer werkt dan is 
de specificatie definitief betrouwbaar. Het tweede punt was mobiliteit. In bovenstaande hoofdstukken is veel te lezen dat alles in te stellen is.
Ook als er naar de software gekeken wordt dan is er te zien dat bijna niks hardcoded is. Hierdoor is het erg mobiel. Het derde punt was 
Management en Monitoring. Er is een basisstation gemaakt waarop alle data te zien is en ook terugkoppeling is naar de gebruiker. Ook zijn 
alle instellingen terug te vinden binnen het systeem. Hierdoor is dit punt ook volbracht. het vierde punt was beveiliging waarbij werd aangegeven 
dat er in ieder geval encryptie moet zijn. Alleen encryptie is geimplementeerd en deze werkt ook volgens verwachting. Het laatste punt was 
unieke adressering. Ook dit is juist geimplementeerd door via hardware aansluitingen een ID te maken. 
Hierdoor zijn alle specificaties voldaan.

\section{Aanbevelingen}
Ookal zijn alle specificaties voldaan is het uiteindelijk ontwerp nog niet geoptimaliseerd. In appendix \ref{app:railsAnalysis} is een RAILS 
analyse opgenomen. Hierin is uitgekomen dat de hardware die was geleverd niet helemaal gewenst is en dat het encryptie algoritme toch iets 
simpler is dan dat er was gehoopt. Hierdoor is het handig dat als er een revisie 2 gemaakt gaat worden er gekeken wordt naar  het onderzoeken van 
een betere chip en een ingewikkeldere encryptie om hopelijk alles efficienter en energiezuiniger te maken, maar ook veiliger.
