\section{Aanbevelingen} \label{sec:aanbevelingen}

% Ook al zijn alle specificaties voldaan is het uiteindelijk ontwerp nog niet geoptimaliseerd. In appendix \ref{app:railsAnalysis} is een RAILS 
% analyse opgenomen. Hierin is uitgekomen dat de hardware die was geleverd niet helemaal gewenst is en dat het encryptie algoritme toch iets 
% simpler is dan dat er was gehoopt. Hierdoor is het handig dat als er een revisie 2 gemaakt gaat worden er gekeken wordt naar het onderzoeken van 
% een betere chip en een ingewikkeldere encryptie om hopelijk alles efficienter en energiezuiniger te maken, maar ook veiliger.

Ondanks dat aan alle specificaties is voldaan, is het uiteindelijke ontwerp nog niet geoptimaliseerd. In \autoref{app:railsAnalysis} zijn enkele belangrijke punten beschreven die vallen onder de volgende categorieën:
\begin{itemize}
\item Snelheid van het systeem
\item Efficiëntie
\item Energiebesparing
\end{itemize}
De beveiliging is momenteel erg eenvoudig. Dit kan verbeterd worden door een geavanceerder beveiligingsalgoritme te gebruiken. De efficiëntie van het algoritme voor het bijhouden van vriendjes kan verbeterd worden door deze te herzien. Daarnaast kan er energie bespaard worden door slaapstanden te gebruiken. Hiernaast is het aan te raden om daadwerkelijke sensoren te gebruiken, in plaats van placeholder potentiometers.

Dit zijn de aanbevelingen voor de volgende ontwerpcyclus voor het ontwikkelen van dit proof of concept.