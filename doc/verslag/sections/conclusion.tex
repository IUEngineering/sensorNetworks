\section{Conclusie} \label{sec:conclusie}

% Het netwerk is nu gemaakt. Als eerste is het belangrijk om te kijken of de specificaties die zijn gemaakt ook zijn volbracht.
% Het eerste punt dat aanwezig moest zijn in het netwerk is betrouwbaarheid. Dit is lastig om te meten, maar wel is er tijdens het ontwerpen 
% gekeken naar hoevaak het programma gereset moest worden om een keer te werken. Dit is niet vaak voorgekomen. Hierdoor kan er gezegd worden dat 
% het hoogswaarschijnlijk betrouwbaar is. De echte test zal zijn tijdens de demo die gegeven gaat worden. Als het dan in één keer werkt dan is 
% de specificatie definitief betrouwbaar. Het tweede punt was mobiliteit. In bovenstaande hoofdstukken is veel te lezen dat alles in te stellen is.
% Ook als er naar de software gekeken wordt dan is er te zien dat bijna niks hardcoded is. Hierdoor is het erg mobiel. Het derde punt was 
% Management en Monitoring. Er is een basisstation gemaakt waarop alle data te zien is en ook terugkoppeling is naar de gebruiker. Ook zijn 
% alle instellingen terug te vinden binnen het systeem. Hierdoor is dit punt ook volbracht. Het vierde punt was beveiliging waarbij werd aangegeven 
% dat er in ieder geval encryptie moet zijn. Alleen encryptie is geimplementeerd en deze werkt ook volgens verwachting. Het laatste punt was 
% unieke adressering. Ook dit is juist geimplementeerd door via hardware aansluitingen een ID te maken. 
% Hierdoor zijn alle specificaties voldaan.

% Het netwerk is nu operationeel, en het is van essentieel belang om te controleren of alle specificaties zijn behaald. Terugkijkend naar paragraaf ... is te lezen dat de specificaties bestonden uit betrouwbaarheid, mobiliteit, management en monitoring, beveiliging en unieke adressering.

% Als eerste betrouwbaarheid: de resultaten in \autoref{sec:tests} tonen aan dat de beste betrouwbaarheid wordt bereikt bij -12dBm en een afstand tussen de nodes van 10 cm en 20 cm. Daarna daalt de betrouwbaarheid naar onder de 50\%. Wat betreft energiebesparing is de betrouwbaarheid het hoogst bij een afstand van 15 cm, maar de ontvangen berichten waren ook acceptabel bij 30 cm. Daarna werden er geen berichten meer ontvangen. Bij de hops-test valt op dat de oriëntatie van de XMega ook veel invloed heeft op de hoeveelheid ontvangen berichten. Hieruit kan worden geconcludeerd dat de betrouwbaarheid van dit systeem afhankelijk is van de oriëntatie van de XMega's, en een afstand van 20 cm lijkt hier het meest ideaal.

% Als tweede is mobiliteit aan de beurt. Dit systeem is zowel voor de PI als voor de sensoren te gebruiken. Het maakt allemaal gebruik van dezelfde software, waarin door middel van hardware toevoegingen onderscheid wordt gemaakt tussen de programma's. Hierdoor is het makkelijk te verplaatsen en kan gesteld worden dat het mobiel is.

% Daarna komt management en monitoring. In het ontwerp is te zien dat er veel gebruik wordt gemaakt van arrays waarin alles wordt opgeslagen. Deze informatie is vervolgens goed terug te vinden door op de Pi te kijken of via de terminal.

% Beveiliging was ook een belangrijk punt. Voor dit punt is een lichte encryptie gekozen. Deze encryptie werkt binnen dit systeem en zal ervoor zorgen dat data niet makkelijk te verspreiden is.

% De laatste specificatie is unieke adressering. Deze werd gedaan door middel van een unieke ID. Dit stond ook gespecificeerd in de ISO. Deze is dan ook hetzelfde over de andere netwerken. In dit netwerk werden de unieke ID's bepaald door hardwarepoorten laag te maken.

% Hierdoor kan worden gezegd dat alle specificaties zijn gehaald. Sommige specificaties zijn echter niet op optimaal niveau gehaald. Verbeterpunten zullen ook te lezen zijn in \autoref{sec:aanbevelingen}.

In dit project is er gewerkt aan een proof of concept, dat de productiviteit van mensen moet verbeteren. In deze eerste ontwerp cyclus is er voornamelijk gewerkt aan het ontwikkelen van een dynamisch draadloos sensornetwerk (DWSN). Dit DWSN bevindt zich nu in een staat dat berichten tussen nodes kunnen worden verstuurd. Deze nodes hoeven niet in directe communicatie te staan, aangezien berichten via andere nodes kunnen worden doorgestuurd. In \autoref{sec:tests} is het aangetoond met metingen dat zolang de afstand tussen nodes niet te groot is, de huidige implementatie van het DWSN een betrouwbaarheid van boven de 80\% kan behalen. 

Naast het ontwikkelen en implementeren wan het DWSN, is er ook gewerkt aan een gebruikersinterface die informatie terug kan geven aan een gebruiker. Deze gebruikersinterface toont de gemeten waardes van de sensoren in het netwerk. Daarnaast toont de gebruikersinterface ook mogelijke metaconclusies. Deze metaconclusies kunnen getrokken worden uit de data die gemeten wordt door de sensoren die op het netwerk zijn aangesloten.

