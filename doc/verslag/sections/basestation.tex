\subsection{Basisstation} \label{sec:basestation}
De basisstation is gemaakt om data weer te geven aan de gebruiker en ook voor het team om debug data te zien. 
\subsubsection{Functionele blokken}

\paragraph{Callback functies}
De gehele interface is zo modulair mogelijk opgebouwt. De windows bestaan uit elementen die je kan aanmaken met de volgende functie.
In deze functie kan je als argument 2 andere fucnties meegeven, een clickCallback die de touch cordinaten mee krijgt en een initCallback die de window meekrijgt zodat je erin de visuele ncruses fucnties kan oproepen. 
\begin{lstlisting}[caption={ScreenElement},captionpos=b,label={lst:ScreenELement},style=c]
static screenElement_t addScreenElement(
    screen_t *screen, uint32_t startRow,
    uint32_t startCol, uint8_t height, uint8_t width,
    void (*clickCallback)(uint32_t, uint32_t),
    void (*initCallback)(WINDOW *win)
); return res - offset;
\end{lstlisting}



\subsubsection{Debug data scherm}
Dit scherm was als eeste gemaakt op de basisstation. De debug scherm was heel erg nuttig om te zien of nodes en het netwerk goed werkten. 
We hebben het gehele scherm gebruikt verschillende data te laten zien. 
\begin{figure}[h]
    \centering
    \includegraphics*[scale=0.17]{img/debugScherm.jpg}
    \caption{Debug scerm op de 7 inch PI scherm}
\end{figure}
Zoals te zien in Figuur 1, is het scherm opgedeelt in 5 delen. Links boven kan je alle broadcast berichten zien die de basisstation ontvangt. 
Links onder in het scherm zie je alle prive relays en payloads die via de basisstation. Dan kijken we rechts boven, hier zitten 2 knoppen en een touch coords window. Een van de knoppen zet de PI veilig uit. De andere switcht naar de User data scherm die bescherven wordt in \ref{userParagraph}.
De coords geven aan waar de laatste touch was op het scherm in rijen. In totaal zijn er 30 rijen op Y as en 100 rijden op de X as. Dit was een van de eerste windows die we hadden toegevoegd en heeft veel geholpen met positioneren van alle objecten op het scherm.
Rechts midden kan je een vriendelijst zien die vanuit de bassistaion xMega wordt weergegeven. Daaronder zien we een `data from sensor nodes', dit is een window die onze dummy data laat zien vanuit de potmeters van andere nodes. 
\subsubsection{User data scherm}\label{userParagraph}
Als je switcht naar de user data scherm dan krijg je de `data from sensor nodes' te zien. Dit is een window precies zoals hij te zien in de debug window.
Ook komt er nog een metaconcluses als de data vanuit de sensoren boven een grens komt. Dan zegt de interface `Je moet een raam open doen, de luchkwaliteit is slecht". Dit is nog iets wat verbeterd moet worden, onze user scherm is heel erg kaal. 
