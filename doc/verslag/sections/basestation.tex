\section{Basisstation} \label{sec:basestation}

% Het basisstation is gemaakt om data weer te geven aan de gebruiker en ook om het communicatiesysteem te kunnen debuggen.
% \subsubsection{Functionele blokken}

% \paragraph{Callback functies}
% De gehele interface is zo modulair mogelijk opgebouwd. De windows bestaan uit elementen die je kan aanmaken met de volgende functie.
% In deze functie kan je als argument 2 andere functies meegeven, een clickCallback die de touch coördinaten mee krijgt en een initCallback die de window meekrijgt zodat je erin de visuele ncruses functies kan oproepen. 
% \begin{lstlisting}[caption={ScreenElement},captionpos=b,label={lst:ScreenELement},style=c]
% static screenElement_t addScreenElement(
%     screen_t *screen, uint32_t startRow,
%     uint32_t startCol, uint8_t height, uint8_t width,
%     void (*clickCallback)(uint32_t, uint32_t),
%     void (*initCallback)(WINDOW *win)
% ); return res - offset;
% \end{lstlisting}



\subsubsection{Debug data scherm}
Het debug scherm was heel erg nuttig om te zien of nodes en het netwerk goed werkten. 
Het gehele scherm is gebruikt verschillende data te laten zien. 
\begin{figure}[h]
    \centering
    \includegraphics*[scale=0.17]{img/debugScherm.jpg}
    \caption{Debug scherm op de 7 inch PI scherm !!DIT MOET EEN SS WORDEN!!}
    \label{fig:test}
\end{figure}
Zoals te zien in \autoref{fig:test}, is het scherm opgedeeld in 5 delen. Links boven kan je alle broadcast berichten zien die de basisstation ontvangt. 
Links onder in het scherm zie je alle privé relays en payloads die via de basisstation. Dan kijken we rechts boven, hier zitten 2 knoppen en een touch coords window. Een van de knoppen zet de PI veilig uit. De andere switcht naar de User data scherm die beschreven wordt in \autoref{ch:userParagraph}.
De coords geven aan waar de laatste touch was op het scherm in rijen en kolommen. In totaal zijn er 30 rijen op Y as en 100 kolommen op de X as. Dit was een van de eerste windows die we hadden toegevoegd en heeft veel geholpen met positioneren van alle objecten op het scherm.
Rechts midden kan je een vriendenlijst zien die vanuit de basisstation xMega wordt weergegeven. Daaronder zien we een `data from sensor nodes', dit is een window die onze dummy data laat zien vanuit de potmeters van andere nodes. 
\subsubsection{User data scherm}\label{ch:userParagraph}
Als je switcht naar de user data scherm dan krijg je de `data from sensor nodes' te zien. Dit is een window precies zoals hij te zien in de debug window.
Ook komt er nog een meta conclusies als de data vanuit de sensoren boven een grens komt. Dan zegt de interface `Je moet een raam open doen, de luchtkwaliteit is slecht". Dit is nog iets wat verbeterd moet worden, onze user scherm is heel erg kaal.