\section{Communicatie} \label{sec:communication}
In dit hoofdstuk wordt ingegaan op hoe de ISO die in het vak sensor netwerken gemaakt is, is geïmplementeerd door team 2. De implementatie is in twee delen opgesplitst: iso en vriendjes lijst. Iso wordt gebruikt om de nrf24l01p aan te sturen terwijl de vriendjes lijst is een erg eenvoudige database voor in een microcontroller is.
\subsection{ISO}
    \subsubsection{Init}
    \subsubsection{Vertrouwen}
    \subsubsection{Update}
    \subsubsection{ID}
    \subsubsection{Stuur pakket}
    
\subsection{Vriendjes}
    In friendList.h en friendList.c worden alle vriendjes (nodes) van het netwerk opgeslagen. 
    \subsubsection{Datastructuur}
        Vriendjes hebben een aantal eigenschappen die onthouden moeten worden om er voor te zorgen dat iso goed kan werken. Een van deze eigenschappen is het unieke ID van een vriendje, hoeveel vriendjes tussen jou en het vriendje in zitten, via wie dit vriendje te bereiken is, hoeveel vertrouwen we hebben in het vriendje en of we het vriendje actief vertrouwen. Omdat deze verschillende elementen vaak nodig zijn is er voor gekozen om deze elementen in een struct samen te voegen en voor programmeer gemak een typedef van deze struct te maken (zie \autoref{lst:friendType}). 
\begin{lstlisting}[caption={Vriendjes struct},captionpos=b,label={lst:friendType},style=c]
typedef struct {
    uint8_t id;
    uint8_t hops;
    uint8_t via;
    uint8_t trust;
    uint8_t active;
}friend_t;
\end{lstlisting}

        Tijdens de initialisatie wordt er een array van 8 friend\_t elementen gealloceerd waar alle id's van de array op 0x00 worden gezet. Deze array is alleen vanuit het friendsList.c bestand te bereiken.

    \subsubsection{Vriendjes toevoegen}
        Om een vriendje toe te voegen wordt de updateFriend functie gebruikt. Deze functie heeft het id, aantal hops en de via data van het nieuwe vriendje nodig. Het id wordt gebruikt om te kijken of het vriendje al bekend is, als het vriendje al bekend is en een directe verbinding heeft (hops = 0) worden de volgende stappen uitgevoerd (zie \autoref{lst:updateDirectFriend}):
        \begin{itemize}
            \item Hoog trust op met TRUST\_ADDER
            \item Als de trust groter is dan MAX\_TRUST zet trust op MAX\_TRUST
            \item Als de trust hoog genoeg is maak het vriendje actief
        \end{itemize}
\begin{lstlisting}[caption={Update direct vriendje},captionpos=b,label={lst:updateDirectFriend},style=c,xleftmargin=.\textwidth,xrightmargin=.\textwidth]
// If it's a direct connection:
if(hops == 0) {
    oldFriend->trust += TRUST_ADDER;

    // Make sure the friend doesn't get too trusted (we have trustissues).
    if(oldFriend->trust > MAX_TRUST) oldFriend->trust = MAX_TRUST;

    // Activate the friend when we trust it enough.
    if(oldFriend->trust > ACTIVATE_TRUST) oldFriend->active = 1;

    DEBUG_PRINTF("\tIt's direct. Increased trust: %2d\tActive: %1d\[0m", \ 
        oldFriend->trust, oldFriend->active);
}
\end{lstlisting}
        Als het vriendje wel bekend is maar niet een direct vriendje is wordt er gekeken of het bestaande vriendje meer hops heeft dan dit nieuwe vriendje. Als dit het geval is worden de via en hops van het oude vriendje vervangen met de via en hops van het nieuwe vriendje (zie \autoref{lst:updateIndirectFriend}). 
\begin{lstlisting}[caption={Update indirect vriendje},captionpos=b,label={lst:updateIndirectFriend},style=c,xleftmargin=.\textwidth,xrightmargin=.\textwidth]
if(hops < oldFriend->hops || oldFriend->hops == 0) {
    oldFriend->via = via;
    oldFriend->hops = hops;
    DEBUG_PRINT("\tNot direct. Replacing hops and via.\e[0m\n");
}
\end{lstlisting}
        In het geval geen van de situaties die hierboven beschreven het geval zijn zal het nieuwe vriendje niet worden opgeslagen.

    \subsubsection{Vriendjes verwijderen}
        Om een vriendje te verwijdere is er een functie gemaakt waar het friend\_t object aan moet worden meegegeven. Als het friend\_t object een ID heeft anders dan NULL zal het aantal vrienden met 1 omlaag worden gehaald. De implementatie hiervan is te zien in \autoref{lst:verwijderFriend}.
        \begin{lstlisting}[caption={Verwijder vriendje},captionpos=b,label={lst:verwijderFriend},style=c,xleftmargin=.\textwidth,xrightmargin=.\textwidth]
void removeFriend(friend_t *friend) {
    friendAmount -= (friend->id != 0);
    friend->id = 0;
}
        \end{lstlisting}

    \subsubsection{Vind vriendje}
        Om te checken of een vriend bekend is met een ID kan de findFriend functie gebruikt worden. Deze functie checkt eerst of het opgegeven ID een valide waarde heeft, in het geval dit niet is zal de functie NULL returnen. Als het meegegeven ID wel valide is wordt stapsgewijs de hele friend\_t array doorgezocht tot of het ID gevonden wordt of tot er het ID niet is gevonden in de hele friend\_t array. In het geval er niet een friend\_t element in de array is gevonden met het opgegeven ID returned de functie NULL. Op het moment dat er wel een friend\_t element in de array is gevonden met het opgegeven ID returned de functie een pointer naar dit friend\_t object. De implementatie hiervan is te zien in \autoref{lst:vindFriend}.
        \begin{lstlisting}[caption={Vind vriendje},captionpos=b,label={lst:vindFriend},style=c,xleftmargin=.\textwidth,xrightmargin=.\textwidth]
// Find a friend, and return a pointer to that friend.
friend_t *findFriend(uint8_t id) {
    // ID 0x00 is invalid.   
    if(id == 0x00) return NULL;

    for(uint8_t i = 0; i < friendListLength; i++) {
        if(friends[i].id == id) return friends + i;
    }
    // If the friend was not found :(
    return NULL;
}
        \end{lstlisting}


\subsection{Encryptie}
Bij een netwerk is het ook handig om een encryptie methode te gebruiken om data van het eigen netwerk veilig te stellen.
Voor dit netwerk is er gekozen om hoe dan ook voor de prive berichten een encryptie te gebruiken omdat dit ervoor zorgt 
dat de data die van het eigen netwerk afkomt normaal zal zijn. Dit is anders als er data naar dit netwerk verstuurt wordt. De 
data die dan ontvangen wordt is onlogisch en kan hierdoor gefilterd worden. 

Tijdens de lessen is er gesproken over 3 verschillende encryptie algoritmes. DES (Data Encryption Standard), AES (Advanced Encryption Standard)
en XOR. Er is uiteindelijk gekozen voor de XOR algoritme omdat deze in een korte tijd goed te implementeren was. Daarentegen is er wel voor
een variatie op de XOR algortime gekozen. Deze heeft na zoeken geen officiele naam en is ook zelf bedacht.
Het werkt alsvolgt:

Wanneer er een bericht binnen komt of wordt verstuurd wordt deze meteen in de keysEncrypt functie gezet. In onderstaande lstlisting \ref{lst:keyEncrypt} 
staat  de keysEncrypt functie beschreven. Op regel 2 is te zien welke soort data er nodig is. de functie moet weten wat message, key1 en
key2 is en wat de lengte van deze 3 zijn. Hierna is er een for loop die evenlang meegaat als dat de lengte van message is. In deze 
for-loop staat dat voor elke even plekje in de array key 1 gebruikt wordt en voor elk oneven plekje wordt key 2 gebruikt. Welke plek 
van de key array gebruikt wordt wordt bepaald door de lengte van de key en de plek van de message. Zo gebruikt bijvoorbeeld plek 0 
sowieso plek 0 van key 1. Als key 1 maar 3 plekken bevat dan gebruikt plek 6 ook plek 0 van key 1. 

\begin{lstlisting}[caption={keyEncrypt Functie},captionpos=b,label={lst:keyEncrypt},style=c,xleftmargin=.\textwidth,xrightmargin=.\textwidth]
    // Encryption and Decryption using multiple key arrays with different lengths
    uint8_t *keysEncrypt(uint8_t* message, uint8_t length, char* key1, 
        uint8_t keyLength1, char * key2, uint8_t keyLength2) {
        static uint8_t data[32];
        for (uint8_t i = 0; i < length; i++) {
            if (i % 2 == 0) {  
                data[i] = (message[i] ^ key1[(i/2) % keyLength1]);
            }
            else {
                data[i] = (message[i] ^ key2[(i/2) % keyLength2]);
            }
        }
        return data;
    }
\end{lstlisting}

De keys zijn variable en niet hardcoded in de code. Hoe dit werkt wordt in hoofstuk \ref{sec:debugProgram} besproken. 
Er is gekozen voor niet hardcoded keys omdat dit ervoor zorgt dat elke keer wanneer het systeem opstart er een nieuwe key bedacht 
kan worden. Hierdoor is de kans op kraken verkleind. De keys kunnen zelfs aangepast worden terwijl het programma nog aanstaat.         