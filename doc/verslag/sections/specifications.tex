\section{Specificaties}

Er zijn voor dit ontwerp 2 verschillende specificaties om op te stellen. De eerste gaat over de sensor data opzichzelf en de andere gaat over hoe 
het netwerk moet werken. Deze twee worden opgesplits omdat dit twee specificaties zijn van het uiteindelijke onderwerp. Alleen hebben deze specificaties 
ook specificaties.

\subsection{Sensor data}
In hoofdstuk \ref{sec:influenceOnComfort} zijn een aantal factoren aan bod gekomen die invloed hebben op het comfort in een kamer.
Deze factoren hebben ook allemaal eenheden. Deze eenheden zijn standaard eenheden per factor. In onderstaande tabel \ref{tabel:meetbereik} staan deze beschreven.
Naast deze eenheden is het ook belangrijk om te weten wanneer een bepaalde factor niet meer gezond is. Hierdoor zal er per eenheid in dezelfde 
tabel \ref{tabel:meetbereik} ook het meetbereik staan.

\begin{table}[h]
    \centering
    \begin{tabular}{|l|l|l|}\hline
        Factor           & Meetbereik            & Eenheid     \\\hline
        Temperatuur      & 10 tot 29 $^{\circ}$C & $^{\circ}$C \\\hline
        Luchtvochtigheid & 0 tot 100\%           & \%          \\\hline
        luchtkwaliteit   & Rood, Groen of Oranje &             \\\hline
        Geluid           & 0 - 90dB              & dB          \\\hline
        Licht            & 0 - 1600Lm            & Lumen       \\\hline
    \end{tabular}
    \caption{Meetbereik en eenheden voor verschillende comfort factoren.}
    \label{tabel:meetbereik}
\end{table}

Een aantal van deze factoren kan snel dan wel langzaam veranderen waardoor ze niet even vaak hoeven worden gemeten. Ook is het goed
 om te weten in welke eenheid deze grootheden gemeten worden. In tabel \ref{tabel:meetfrequentie} is de meetfrequentie en eenheid per 
 factor beschreven.

\begin{table}[h]
    \centering
    \begin{tabular}{|l|l|}\hline
        Factor           & Meetfrequentie\\\hline
        Temperatuur      & 30 min        \\\hline
        Luchtvochtigheid & 30 min        \\\hline
        luchtkwaliteit   & 10 min        \\\hline
        Geluid           & 5 sec         \\\hline
        Licht            & 10 sec        \\\hline
    \end{tabular}
    \caption{Meetfrequentie voor verschillende comfort factoren.}
    \label{tabel:meetfrequentie}
\end{table}

\subsection{Netwerk}
In hoofdstuk \ref{sec:theory} is er naast sensor data ook netwerken besproken. Hieronder een lijst met specificaties die daaruit zijn 
opgesteld en die volgens de opdrachtgever belangrijk zijn om mee te nemen.
\begin{itemize}
    \item Betrouwbaarheid
    \item Mobiliteit
    \item Management en Monitoring
    \item Beveiliging
    \item Unieke adressering
\end{itemize}

De opdrachtgever heeft al aangegeven dat voor mobiliteit het een must is om alle nodes draadloos met elkaar en het basisstation te laten communiceren.
Voor betrouwbaarheid is het belangrijk dat het systeem zo min mogelijk opnieuw opgestart moet worden om te werken.
Er moet een basisstation zijn waarop alle data zichtbaar en overzichtelijk leesbaar is. 
Voor de beveiliging is alleen encryptie een must. Alles dat daarna wordt gedaan is mooi meegenomen.
De opdrachtgever heeft ook aangegeven dat er een unieke adressering moet zijn.