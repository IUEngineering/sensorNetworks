\section{Ontwerp}
In de aankomende paragraven zal de focus liggen op het ontwerpen van het netwerk. Naast de sensoren ontwikkelen is dit een erg groot gedeelte
van het eindproduct. Omdat er wel al duidelijk is wat voor data er binnen zal komen, zal deze data voor nu gedaan worden doormiddel van dummydata. 
Deze data zal simuleren wat voor echte data er later binnen zal komen. Vanuit de theorie en specificaties zullen de grenzen van goed en niet goed wel overgenomen worden 
in het ontwerp om ervoor te zorgen dat de juiste conclusies op het basis station te zien zijn. 

De eerste keuze die is gemaakt bij het maken van het netwerk is om software te bedenken die door hardware aansluitingen bepaald welk programma het is.
Dit zorgt ervoor dat er maar 1 software geschreven hoeft te worden. Het voordelen die hierbij komen zijn structuur, modulairiteit en indirect zorgt dit ook voor 
tijd besparing. Dit komt omdat functies die eventueel gebruikt gaat worden tussen de programma's niet opnieuw geschreven hoeft te worden.

\section{Communicatie} \label{sec:communication}
In dit hoofdstuk wordt ingegaan op hoe de ISO die in het vak sensor netwerken gemaakt is, is geïmplementeerd door team 2. De implementatie is in twee delen opgesplitst: iso en vriendjes lijst. Iso wordt gebruikt om de nrf24l01p aan te sturen terwijl de vriendjes lijst is een erg eenvoudige database voor in een microcontroller is.
\subsection{ISO}
    \subsubsection{Init}
    \subsubsection{Vertrouwen}
    \subsubsection{Update}
    \subsubsection{ID}
    \subsubsection{Stuur pakket}
    
\subsection{Vriendjes}
    In friendList.h en friendList.c worden alle vriendjes (nodes) van het netwerk opgeslagen. 
    \subsubsection{Datastructuur}
        Vriendjes hebben een aantal eigenschappen die onthouden moeten worden om er voor te zorgen dat iso goed kan werken. Een van deze eigenschappen is het unieke ID van een vriendje, hoeveel vriendjes tussen jou en het vriendje in zitten, via wie dit vriendje te bereiken is, hoeveel vertrouwen we hebben in het vriendje en of we het vriendje actief vertrouwen. Omdat deze verschillende elementen vaak nodig zijn is er voor gekozen om deze elementen in een struct samen te voegen en voor programmeer gemak een typedef van deze struct te maken (zie \autoref{lst:friendType}). 
\begin{lstlisting}[caption={Vriendjes struct},captionpos=b,label={lst:friendType},style=c]
typedef struct {
    uint8_t id;
    uint8_t hops;
    uint8_t via;
    uint8_t trust;
    uint8_t active;
}friend_t;
\end{lstlisting}

        Tijdens de initialisatie wordt er een array van 8 friend\_t elementen gealloceerd waar alle id's van de array op 0x00 worden gezet. Deze array is alleen vanuit het friendsList.c bestand te bereiken.

    \subsubsection{Vriendjes toevoegen}
        Om een vriendje toe te voegen wordt de updateFriend functie gebruikt. Deze functie heeft het id, aantal hops en de via data van het nieuwe vriendje nodig. Het id wordt gebruikt om te kijken of het vriendje al bekend is, als het vriendje al bekend is en een directe verbinding heeft (hops = 0) worden de volgende stappen uitgevoerd (zie \autoref{lst:updateDirectFriend}):
        \begin{itemize}
            \item Hoog trust op met TRUST\_ADDER
            \item Als de trust groter is dan MAX\_TRUST zet trust op MAX\_TRUST
            \item Als de trust hoog genoeg is maak het vriendje actief
        \end{itemize}
\begin{lstlisting}[caption={Update direct vriendje},captionpos=b,label={lst:updateDirectFriend},style=c,xleftmargin=.\textwidth,xrightmargin=.\textwidth]
// If it's a direct connection:
if(hops == 0) {
    oldFriend->trust += TRUST_ADDER;

    // Make sure the friend doesn't get too trusted (we have trustissues).
    if(oldFriend->trust > MAX_TRUST) oldFriend->trust = MAX_TRUST;

    // Activate the friend when we trust it enough.
    if(oldFriend->trust > ACTIVATE_TRUST) oldFriend->active = 1;

    DEBUG_PRINTF("\tIt's direct. Increased trust: %2d\tActive: %1d\[0m", \ 
        oldFriend->trust, oldFriend->active);
}
\end{lstlisting}
        Als het vriendje wel bekend is maar niet een direct vriendje is wordt er gekeken of het bestaande vriendje meer hops heeft dan dit nieuwe vriendje. Als dit het geval is worden de via en hops van het oude vriendje vervangen met de via en hops van het nieuwe vriendje (zie \autoref{lst:updateIndirectFriend}). 
\begin{lstlisting}[caption={Update indirect vriendje},captionpos=b,label={lst:updateIndirectFriend},style=c,xleftmargin=.\textwidth,xrightmargin=.\textwidth]
if(hops < oldFriend->hops || oldFriend->hops == 0) {
    oldFriend->via = via;
    oldFriend->hops = hops;
    DEBUG_PRINT("\tNot direct. Replacing hops and via.\e[0m\n");
}
\end{lstlisting}
        In het geval geen van de situaties die hierboven beschreven het geval zijn zal het nieuwe vriendje niet worden opgeslagen.

    \subsubsection{Vriendjes verwijderen}
        Om een vriendje te verwijdere is er een functie gemaakt waar het friend\_t object aan moet worden meegegeven. Als het friend\_t object een ID heeft anders dan NULL zal het aantal vrienden met 1 omlaag worden gehaald. De implementatie hiervan is te zien in \autoref{lst:verwijderFriend}.
        \begin{lstlisting}[caption={Verwijder vriendje},captionpos=b,label={lst:verwijderFriend},style=c,xleftmargin=.\textwidth,xrightmargin=.\textwidth]
void removeFriend(friend_t *friend) {
    friendAmount -= (friend->id != 0);
    friend->id = 0;
}
        \end{lstlisting}

    \subsubsection{Vind vriendje}
        Om te checken of een vriend bekend is met een ID kan de findFriend functie gebruikt worden. Deze functie checkt eerst of het opgegeven ID een valide waarde heeft, in het geval dit niet is zal de functie NULL returnen. Als het meegegeven ID wel valide is wordt stapsgewijs de hele friend\_t array doorgezocht tot of het ID gevonden wordt of tot er het ID niet is gevonden in de hele friend\_t array. In het geval er niet een friend\_t element in de array is gevonden met het opgegeven ID returned de functie NULL. Op het moment dat er wel een friend\_t element in de array is gevonden met het opgegeven ID returned de functie een pointer naar dit friend\_t object. De implementatie hiervan is te zien in \autoref{lst:vindFriend}.
        \begin{lstlisting}[caption={Vind vriendje},captionpos=b,label={lst:vindFriend},style=c,xleftmargin=.\textwidth,xrightmargin=.\textwidth]
// Find a friend, and return a pointer to that friend.
friend_t *findFriend(uint8_t id) {
    // ID 0x00 is invalid.   
    if(id == 0x00) return NULL;

    for(uint8_t i = 0; i < friendListLength; i++) {
        if(friends[i].id == id) return friends + i;
    }
    // If the friend was not found :(
    return NULL;
}
        \end{lstlisting}


\subsection{Encryptie}
Bij een netwerk is het ook handig om een encryptie methode te gebruiken om data van het eigen netwerk veilig te stellen.
Voor dit netwerk is er gekozen om hoe dan ook voor de prive berichten een encryptie te gebruiken omdat dit ervoor zorgt 
dat de data die van het eigen netwerk afkomt normaal zal zijn. Dit is anders als er data naar dit netwerk verstuurt wordt. De 
data die dan ontvangen wordt is onlogisch en kan hierdoor gefilterd worden. 

Tijdens de lessen is er gesproken over 3 verschillende encryptie algoritmes. DES (Data Encryption Standard), AES (Advanced Encryption Standard)
en XOR. Er is uiteindelijk gekozen voor de XOR algoritme omdat deze in een korte tijd goed te implementeren was. Daarentegen is er wel voor
een variatie op de XOR algortime gekozen. Deze heeft na zoeken geen officiele naam en is ook zelf bedacht.
Het werkt alsvolgt:

Wanneer er een bericht binnen komt of wordt verstuurd wordt deze meteen in de keysEncrypt functie gezet. In onderstaande lstlisting \ref{lst:keyEncrypt} 
staat  de keysEncrypt functie beschreven. Op regel 2 is te zien welke soort data er nodig is. de functie moet weten wat message, key1 en
key2 is en wat de lengte van deze 3 zijn. Hierna is er een for loop die evenlang meegaat als dat de lengte van message is. In deze 
for-loop staat dat voor elke even plekje in de array key 1 gebruikt wordt en voor elk oneven plekje wordt key 2 gebruikt. Welke plek 
van de key array gebruikt wordt wordt bepaald door de lengte van de key en de plek van de message. Zo gebruikt bijvoorbeeld plek 0 
sowieso plek 0 van key 1. Als key 1 maar 3 plekken bevat dan gebruikt plek 6 ook plek 0 van key 1. 

\begin{lstlisting}[caption={keyEncrypt Functie},captionpos=b,label={lst:keyEncrypt},style=c,xleftmargin=.\textwidth,xrightmargin=.\textwidth]
    // Encryption and Decryption using multiple key arrays with different lengths
    uint8_t *keysEncrypt(uint8_t* message, uint8_t length, char* key1, 
        uint8_t keyLength1, char * key2, uint8_t keyLength2) {
        static uint8_t data[32];
        for (uint8_t i = 0; i < length; i++) {
            if (i % 2 == 0) {  
                data[i] = (message[i] ^ key1[(i/2) % keyLength1]);
            }
            else {
                data[i] = (message[i] ^ key2[(i/2) % keyLength2]);
            }
        }
        return data;
    }
\end{lstlisting}

De keys zijn variable en niet hardcoded in de code. Hoe dit werkt wordt in hoofstuk \ref{sec:debugProgram} besproken. 
Er is gekozen voor niet hardcoded keys omdat dit ervoor zorgt dat elke keer wanneer het systeem opstart er een nieuwe key bedacht 
kan worden. Hierdoor is de kans op kraken verkleind. De keys kunnen zelfs aangepast worden terwijl het programma nog aanstaat.         
\section{Dummy data} \label{sec:dummyData}

% Het dummy data programma maakt het mogelijk om het dynamische draadloze sensor netwerk te testen. In Hoofdstuk \ref{sec:influenceOnComfort} staat beschreven welke soorten data er vanuit een sensor node moeten kunnen worden verstuurd. Omdat het systeem met meerdere sensor nodes moet worden getest zal niet elke sensor node alle sensor data genereren indien dit gewenst is. 

% \subsubsection{Nep data}
% De nep data wordt gegenereerd door een spanning aan te bieden op een van de pinnen A0 - A4, deze spanning moet tussen 0V en 2V zijn. De nep data spanning wordt van het analoge domein naar het digitale domein omgezet door een ADC. Omdat ADC's vaak een offset hebben wordt pin A5 met GND verbonden om de offset te meten en dit van de meet resultaten af te halen. De ADC conversies worden gestart op het moment dat de nep data nodig is voor het verzenden van informatie.

% \subsubsection{ADC}
% De ADC is ingesteld op single ended signalen omdat er alleen maar positive signalen worden verwacht. De overige instellingen zijn te zien in \autoref{lst:adcInit}.
% \begin{lstlisting}[caption={ADC init},captionpos=b,label={lst:adcInit},style=c]
% // congigure ADCA
% ADCA.REFCTRL     = ADC_REFSEL_INTVCC_gc;
% ADCA.CTRLB       = ADC_RESOLUTION_12BIT_gc;            
% ADCA.PRESCALER   = ADC_PRESCALER_DIV16_gc;

% // Configure input channels
% PORTA.DIRCLR     =  PIN0_bm |
%                     PIN1_bm |
%                     PIN2_bm | 
%                     PIN3_bm |
%                     PIN4_bm |
%                     PIN5_bm;

% ADCA.CH0.CTRL    = ADC_CH_INPUTMODE_SINGLEENDED_gc;

% // Enable the ADC
% ADCA.CTRLA       = ADC_ENABLE_bm;
% \end{lstlisting}
% Als er nep data voor een sensor moet worden gegenereerd wordt de functie ADCReadCH0 aangeroepen. ADCReadCH0 heeft 1 argument waarmee aangegeven wordt welke pin op de positive ingang van de ADC moet worden aangesloten via de MUX. Daarnaast wordt er gemeten op PA5 die verbonden is aan GND wat de offset is van de ADC. 
% \begin{lstlisting}[caption={ADCReadCH0},captionpos=b,label={lst:ADCReadCH0},style=c]
% uint16_t ADCReadCH0(uint8_t inputPin) { 
%     uint16_t offset;                                  
%     uint16_t res;

%     // Measure the offset
%     ADCA.CH0.MUXCTRL = ADC_CH_MUXPOS_PIN5_gc;

%     ADCA.CH0.CTRL |= ADC_CH_START_bm;
%     while ( !(ADCA.CH0.INTFLAGS & ADC_CH_CHIF_bm) );
    
%     offset = ADCA.CH0.RES;
%     ADCA.CH0.INTFLAGS |= ADC_CH_CHIF_bm;

%     // Measure the signal
%     ADCA.CH0.MUXCTRL = inputPin;

%     ADCA.CH0.CTRL |= ADC_CH_START_bm;
%     while ( !(ADCA.CH0.INTFLAGS & ADC_CH_CHIF_bm) );

%     res = ADCA.CH0.RES;
%     ADCA.CH0.INTFLAGS |= ADC_CH_CHIF_bm;

%     if (offset > res)
%         res = offset;

%     return res - offset;
% }
% \end{lstlisting}

% \subsubsection{Timing}
% Om de tijd bij te houden wordt gebruik gemaakt van timercounter E0 (TCE0). Het is niet mogelijk om 30 minuten bij te houden op de TCE0 waardoor er is gekozen om elke seconden een variable op te hogen en te kijken welke dat er moet worden verzonden zoals te zien is in \autoref{lst:dummyDataLoop}.
% \begin{lstlisting}[caption={dummyData loop},captionpos=b,label={lst:dummyDataLoop},style=c]
% while(! TCE0.INTFLAGS & TC0_OVFIF_bm)
%     isoUpdate();

% TCE0.INTFLAGS = TC0_OVFIF_bm;
% timer++;

% if ((timer % TIME_5_SEC == 0) && (PORTB.IN & PIN4_bm))
%     sendSound();

% if ((timer % TIME_10_SEC == 0)  && (PORTB.IN & PIN2_bm))
%     sendLight();

% if ((timer % TIME_10_MIN == 0)  && (PORTB.IN & PIN1_bm))
%     sendAirQuality();

% if (timer % TIME_30_MIN == 0) {
%     if (PORTB.IN & PIN3_bm)
%         sendTemp();
    
%     if (PORTB.IN & PIN0_bm)
%         sendAirMoisture();
%     timer = 0;
% }
% \end{lstlisting}
\section{Basisstation} \label{sec:basestation}

% De basisstation is gemaakt om data weer te geven aan de gebruiker en ook voor het team om debug data te zien. 
% \subsubsection{Functionele blokken}

% \paragraph{Callback functies}
% De gehele interface is zo modulair mogelijk opgebouwt. De windows bestaan uit elementen die je kan aanmaken met de volgende functie.
% In deze functie kan je als argument 2 andere fucnties meegeven, een clickCallback die de touch cordinaten mee krijgt en een initCallback die de window meekrijgt zodat je erin de visuele ncruses fucnties kan oproepen. 
% \begin{lstlisting}[caption={ScreenElement},captionpos=b,label={lst:ScreenELement},style=c]
% static screenElement_t addScreenElement(
%     screen_t *screen, uint32_t startRow,
%     uint32_t startCol, uint8_t height, uint8_t width,
%     void (*clickCallback)(uint32_t, uint32_t),
%     void (*initCallback)(WINDOW *win)
% ); return res - offset;
% \end{lstlisting}



% \subsubsection{Debug data scherm}
% Dit scherm was als eeste gemaakt op de basisstation. De debug scherm was heel erg nuttig om te zien of nodes en het netwerk goed werkten. 
% We hebben het gehele scherm gebruikt verschillende data te laten zien. 
% \begin{figure}[h]
%     \centering
%     \includegraphics*[scale=0.17]{img/debugScherm.jpg}
%     \caption{Debug scerm op de 7 inch PI scherm}
% \end{figure}
% Zoals te zien in Figuur 1, is het scherm opgedeelt in 5 delen. Links boven kan je alle broadcast berichten zien die de basisstation ontvangt. 
% Links onder in het scherm zie je alle prive relays en payloads die via de basisstation. Dan kijken we rechts boven, hier zitten 2 knoppen en een touch coords window. Een van de knoppen zet de PI veilig uit. De andere switcht naar de User data scherm die bescherven wordt in \ref{userParagraph}.
% De coords geven aan waar de laatste touch was op het scherm in rijen. In totaal zijn er 30 rijen op Y as en 100 rijden op de X as. Dit was een van de eerste windows die we hadden toegevoegd en heeft veel geholpen met positioneren van alle objecten op het scherm.
% Rechts midden kan je een vriendelijst zien die vanuit de bassistaion xMega wordt weergegeven. Daaronder zien we een `data from sensor nodes', dit is een window die onze dummy data laat zien vanuit de potmeters van andere nodes. 
% \subsubsection{User data scherm}\label{userParagraph}
% Als je switcht naar de user data scherm dan krijg je de `data from sensor nodes' te zien. Dit is een window precies zoals hij te zien in de debug window.
% Ook komt er nog een metaconcluses als de data vanuit de sensoren boven een grens komt. Dan zegt de interface `Je moet een raam open doen, de luchkwaliteit is slecht". Dit is nog iets wat verbeterd moet worden, onze user scherm is heel erg kaal. 

\section{Debug chat terminal}

Het NrfChat programma laat mensen met elkaar praten via een CLI interface. Om het programma in te stellen kunnen een aantal commando's

\subsection{Commands}
De commando's die in het nrfChat programma kunnen worden gebruikt.

\begin{table}[h]
    \begin{tabular}{|l|l|} \hline
        \textbf{Commando} & \textbf{Beschrijving} \\\hline
        help & Laat een hulp bericht zien \\\hline
        chan & Selecteer het kanaal waarop wordt gezonden en ontvangen\\\hline
        send & Stuur een bericht \\\hline
        list & Laat de vriendjes lijst zien\\\hline
        dest & Stel doel ID in voor berichten die met send worden verstuurd \\\hline
        myid & Laat zien wat het ID is van de node \\\hline
    \end{tabular}
\end{table}

\subsection{Functionele blokken}

\subsection{nrfChat}
    \subsubsection{Init}
    \subsubsection{Loop}
    \subsubsection{Intrepeteer input}
    \subsubsection{Help}
    \subsubsection{Chan}
    \subsubsection{Send}
    \subsubsection{List}
    \subsubsection{Dest}
    \subsubsection{MyID}

\subsection{Terminal}
    \subsubsection{Intrepeteer input}
    \subsubsection{Command callback}
    \subsubsection{Terminal print}
    \subsubsection{Terminal print hex}
    \subsubsection{Terminal print strex}
