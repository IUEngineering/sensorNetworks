% o Inhoudelijke verkenning, kennis benodigd voordat met het ontwerp gestart kan worden (o.a. normen en regelgeving), 
% vaak a.d.h.v. gerelateerd werk.
% o Relevante onderzoeksvragen worden hierin uitgewerkt.
% o Welke literatuur en/of theorieën zijn relevant en wat betekent dit voor het
% ontwerp.
% o Overzicht van bestaande oplossingen van het probleem en waarom voldoen
% deze in dit specifieke geval wel/niet

Te lezen in Hoofdstuk \ref{opdracht} is dat er dus een probleem is met comfortabelheid in een ruimte. In de probleemstelling is al 
kort verteld welke grootheden hier invloed op hebben. Dit waren luchtkwaliteit, licht en geluid. Dit zijn ze daarentegen niet allemaal. 
In het onderzoek die is uitgevoerd door Kees Sietsma \cite{productiviteit} is te zien dat hij is gaan kijken naar de 
productiviteit van werknemers binnen een bedrijf. Uit dit onderzoek is gebleken dat werknemers minder productief zijn als de temperatuur
boven de 20 graden is. Bij elke graad staat dat de productiviteit daalt met 4\%. Ook is een werknemer gemiddeld 6\% productiver in een rustige
en stille ruimte dan een werknemer die in een ruimte zit waar veel geluid te horen is en waar veel velle kleuren te zien zijn. \\

Deze grootheden opzichzelf zijn al niet zo fijn, maar combintaties hiervan zullen nog erger zijn. Als er bijvoorbeeld wordt gekeken naar
temperatuur opzichzelf dan kan dat niet comfortabel zijn bij bijvoorbeeld 24 graden, maar iemand zou hier nog gewoon in kunnen werken. 
Als de factor luchtvochtigheid bij komt kijken dan is het ineens een ander verhaal. Dan kan het voor de gebruiker erg benauwd aan kunnen
voelen en is de ruimte hierdoor dus niet meer fijn en productief om in te werken. Bijvoorbeeld de combinatie licht en geluid kan er na 
een tijdje voor zorgen dat iemand hoofdpijn kan krijgen. Luchtkwaliteit en luchtvochtigheid kan ook voor een benauwd gevoel zorgen. \\

Dit zijn goeie grootheden om te gebruiken in het netwerk. Maar de gebruiker moet ook feedback terug krijgen over de waardes die worden
gemeten en eventueel horen of zien dat het niet meer productief is om in de ruimte te blijven. Hiervoor is een basisstation nodig. 
Op dit station zou alle data binnen moeten komen. De gebruiker kan hier dus sowieso de waardes op uitlezen. Alleen is het logisch dat
een gebruiker het verschil tussen 13 of 18 microgram per cubic meter. Wat in dit geval volgens de KNMI \cite{Gezonde} een verschil is tussen
"de luchtkwaliteit is prima" en "de luchtkwaliteit is niet meer gezond". Hierdoor moet er voor elke parameter een kleine waarschuwing krijgen
wanneer een bepaalde waarde is bereikt. Als er bijvoorbeeld wordt gekeken naar een telefoon en hoe deze meldingen laat zien dan zijn er 
verschillende methodes om dat te doen. Dit wordt gedaan door een geluidje om de aandacht te trekken of door het scherm op te lichten. Soms
is alleen dat niet genoeg en gaat de telefoon trillen en er is ook een mogelijkheid om een knipperende scherm melding te krijgen.

%uitleg over elke eenheid van de grootheden




