\section{Debug chat terminal}

Het NrfChat programma laat mensen met elkaar praten via een CLI interface. Om het programma in te stellen kunnen een aantal commando's

\subsection{Commands}
De commando's die in het nrfChat programma kunnen worden gebruikt.

\begin{table}[h]
    \begin{tabular}{|l|l|} \hline
        \textbf{Commando} & \textbf{Beschrijving} \\\hline
        help & Laat een hulp bericht zien \\\hline
        chan & Selecteer het kanaal waarop wordt gezonden en ontvangen\\\hline
        send & Stuur een bericht \\\hline
        list & Laat de vriendjes lijst zien\\\hline
        dest & Stel doel ID in voor berichten die met send worden verstuurd \\\hline
        myid & Laat zien wat het ID is van de node \\\hline
    \end{tabular}
\end{table}

\subsection{Functionele blokken}

\subsection{nrfChat}
    \subsubsection{Init}
    \subsubsection{Loop}
    \subsubsection{Intrepeteer input}
    \subsubsection{Help}
    \subsubsection{Chan}
    \subsubsection{Send}
    \subsubsection{List}
    \subsubsection{Dest}
    \subsubsection{MyID}

\subsection{Terminal}
    \subsubsection{Intrepeteer input}
    \subsubsection{Command callback}
    \subsubsection{Terminal print}
    \subsubsection{Terminal print hex}
    \subsubsection{Terminal print strex}
