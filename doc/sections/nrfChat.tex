\section{Communicatie}

In dit hoofdstuk wordt ingegaan op hoe de ISO die in het vak sensor netwerken gemaakt is, is geïmplementeerd door team 2. De implementatie is in twee delen opgesplitst: iso en vriendjes lijst. Iso wordt gebruikt om de nrf24l01p aan te sturen terwijl de vriendjes lijst is een erg eenvoudige database voor in een microcontroller is.
\subsection{ISO}
    \subsubsection{Init}
    \subsubsection{Vertrouwen}
    \subsubsection{Update}
    \subsubsection{ID}
    \subsubsection{Stuur pakket}
    
\subsection{Vriendjes}
    In friendList.h en friendList.c worden alle vriendjes (nodes) van het netwerk opgeslagen. 
    \subsubsection{Datastructuur}
        Vriendjes hebben een aantal eigenschappen die onthouden moeten worden om er voor te zorgen dat iso goed kan werken. Een van deze eigenschappen is het unieke ID van een vriendje, hoeveel vriendjes tussen jou en het vriendje in zitten, via wie dit vriendje te bereiken is, hoeveel vertrouwen we hebben in het vriendje en of we het vriendje actief vertrouwen. Omdat deze verschillende elementen vaak nodig zijn is er voor gekozen om deze elementen in een struct samen te voegen en voor programmeer gemak een typedef van deze struct te maken (zie \autoref{lst:friendType}). 
\begin{lstlisting}[caption={Vriendjes struct},captionpos=b,label={lst:friendType},style=c]
typedef struct {
    uint8_t id;
    uint8_t hops;
    uint8_t via;
    uint8_t trust;
    uint8_t active;
}friend_t;
\end{lstlisting}

        Tijdens de initialisatie wordt er een array van 8 friend\_t elementen gealloceerd waar alle id's van de array op 0x00 worden gezet. Deze array is alleen vanuit het friendsList.c bestand te bereiken.

    \subsubsection{Vriendjes toevoegen}
        Om een vriendje toe te voegen wordt de updateFriend functie gebruikt. Deze functie heeft het id, aantal hops en de via data van het nieuwe vriendje nodig. Het id wordt gebruikt om te kijken of het vriendje al bekend is, als het vriendje al bekend is en een directe verbinding heeft (hops = 0) worden de volgende stappen uitgevoerd (zie \autoref{lst:updateDirectFriend}):
        \begin{itemize}
            \item Hoog trust op met TRUST\_ADDER
            \item Als de trust groter is dan MAX\_TRUST zet trust op MAX\_TRUST
            \item Als de trust hoog genoeg is maak het vriendje actief
        \end{itemize}
\begin{lstlisting}[caption={Update direct vriendje},captionpos=b,label={lst:updateDirectFriend},style=c,xleftmargin=.\textwidth,xrightmargin=.\textwidth]
// If it's a direct connection:
if(hops == 0) {
    oldFriend->trust += TRUST_ADDER;

    // Make sure the friend doesn't get too trusted (we have trustissues).
    if(oldFriend->trust > MAX_TRUST) oldFriend->trust = MAX_TRUST;

    // Activate the friend when we trust it enough.
    if(oldFriend->trust > ACTIVATE_TRUST) oldFriend->active = 1;

    DEBUG_PRINTF("\tIt's direct. Increased trust: %2d\tActive: %1d\[0m", \ 
        oldFriend->trust, oldFriend->active);
}
\end{lstlisting}
        Als het vriendje wel bekend is maar niet een direct vriendje is wordt er gekeken of het bestaande vriendje meer hops heeft dan dit nieuwe vriendje. Als dit het geval is worden de via en hops van het oude vriendje vervangen met de via en hops van het nieuwe vriendje (zie \autoref{lst:updateIndirectFriend}). 
\begin{lstlisting}[caption={Update indirect vriendje},captionpos=b,label={lst:updateIndirectFriend},style=c,xleftmargin=.\textwidth,xrightmargin=.\textwidth]
if(hops < oldFriend->hops || oldFriend->hops == 0) {
    oldFriend->via = via;
    oldFriend->hops = hops;
    DEBUG_PRINT("\tNot direct. Replacing hops and via.\e[0m\n");
}
\end{lstlisting}
        In het geval geen van de situaties die hierboven zijn beschreven het geval zijn zal het nieuwe vriendje niet worden meegenomen.
    \subsubsection{Vriendjes verwijderen}
    \subsubsection{Vind vriendje}
    \subsubsection{Vriendjes lijst} \label{ch:printFriends}

\section{Base station}
    \subsection{Functionele blokken}
\section{Debug chat terminal}

Het NrfChat programma laat mensen met elkaar praten via een CLI interface. Om het programma in te stellen kunnen een aantal commando's

\subsection{Commands}
De commando's die in het nrfChat programma kunnen worden gebruikt.

\begin{table}[h]
    \begin{tabular}{|l|l|} \hline
        \textbf{Commando} & \textbf{Beschrijving} \\\hline
        help & Laat een hulp bericht zien \\\hline
        chan & Selecteer het kanaal waarop wordt gezonden en ontvangen\\\hline
        send & Stuur een bericht \\\hline
        list & Laat de vriendjes lijst zien\\\hline
        dest & Stel doel ID in voor berichten die met send worden verstuurd \\\hline
        myid & Laat zien wat het ID is van de node \\\hline
    \end{tabular}
\end{table}

\subsection{Functionele blokken}

\subsection{nrfChat}
    \section{Debug nodes}
Het nrfChat bestand is bedoeld om op een gebruikersvriendelijke manier met andere nodes te communiceren. Het programma laat de terminal in principe functioneren als command line, waar commando's in getypt kunnen worden om uit te voeren. Het is niet bedoeld als een sensormodule, maar in plaats daarvan als een nuttige debugging terminal voor engineers.

De gebruiker kan commando's invoeren, beginnende met een slash, die onder andere gebruikt kunnen worden om de bestemming van berichten in te stellen, of de huidige burenlijst/vriendjeslijst uit te printen. Alles wat in de terminal getypt wordt dat niet begint met een slash, wordt gezien als een bericht om verstuurd te worden. Deze berichten worden via iso.c verstuurd naar de geselecteerde ontvanger.

\paragraph*{Init}
Bij het initialiseren van nrfChat worden de volgende acties ondernomen:
\begin{enumerate}
    \item iso.c wordt geïnitialiseerd. 
    \item De ontvangst callback functie van iso.c wordt ingesteld op de ontvangst functie (zie \autoref{ch:messageReceive}).
    \item De gebruikersinput callback functie van terminal.c ingesteld op de input interpreteer functie (zie \autoref{ch:interpreter}).
    \item Er wordt een header geprint met een welkomstbericht met een instructie om /help in te voeren, en het ID en kanaal van de node.
\end{enumerate}

Hierna kan de gebruiker commando's en berichten typen.

\paragraph*{Loop}
In de loop worden ingevoerde karakters doorgegeven aan terminal.c. Ook wordt \texttt{isoUpdate} aangeroepen. Zie \autoref{lst:nrfChatLoop}.

\begin{lstlisting}[caption={De nrfChat loop},captionpos=b,label={lst:nrfChatLoop},style=c,xleftmargin=.\textwidth,xrightmargin=.\textwidth]
void nrfChatLoop(void) {
    while (1) {
        // Get the character from the user.
        // Use char instead of uint16_t,
        // because we don't need to handle any '\0's.
        char newInputChar = uartF0_getc();

        if(newInputChar != '\0') {
           terminalInterpretChar(newInputChar);
        }

        isoUpdate();
    }
}
\end{lstlisting}

\paragraph*{Berichten ontvangen} \label{ch:messageReceive}
Ontvangen berichten worden direct doorgestuurd naar de \texttt{terminalPrintStrex} functie van terminal.c. Zie \autoref{ch:printStrex}.

\paragraph*{Interpreteer input} \label{ch:interpreter}

De input wordt in meerdere stappen geïnterpreteerd. De interpreteer functie krijgt de ingevoerde regel tekst mee als argument, en wordt uitgevoerd wanneer er op enter wordt gedrukt door de gebruiker.
Het eerste deel van de functie is te zien in \autoref{lst:interpretInputStart}. Hier kijkt de functie eerst of de input begint met een slash. Als dit niet het geval is, wordt de volledige input direct verstuurd naar de geselecteerde bestemming. Wanneer de input wel met een slash begint, wordt de pointer naar de input met 1 verhoogt, zodat de rest van de code geen rekening meer hoeft te houden met deze slash.

\begin{lstlisting}[caption={Eerste deel van de interpreter},captionpos=b,label={lst:interpretInputStart},style=c,xleftmargin=.\textwidth,xrightmargin=.\textwidth]
// If no '/' is given, just send the inputted string.
if(input[0] != '/') {
    send(input);
    return;
}

// If a '/' is given, we want to interpret from the character after it.
input++;
\end{lstlisting}

Er worden vervolgens twee constante en statische arrays aangemaakt (zie \autoref{lst:interpretInputFPointers}). Eén van de arrays bevat al de commando's die uitgevoerd kunnen worden door de gebruiker. De andere array bevat de functie pointers naar de corresponderende functies.
Aangezien de functienamen op de zelfde index zitten als de functie pointers, kunnen de arrays gebruikt worden als look-up table.

\begin{lstlisting}[caption={De look-up table van de interpreter},captionpos=b,label={lst:interpretInputFPointers},style=c,xleftmargin=.\textwidth,xrightmargin=.\textwidth]
// Make a corresponding array of 4 letter function names.
static const char commands[COMMANDS][4] = {
    "send", "help", "chan", "list", "dest", "myid", "keys", "key1", "key2"
};
// Make an array of functions.
static const void (*functions[COMMANDS])(char*) = {
    send, help, chan, list, dest, myid, keys, key1, key2
};
\end{lstlisting}

De code gaat vervolgens langs elk van de functienamen. Wanneer het ingevoerde commando gelijk is aan een van de functienamen in de \texttt{commands} array, wordt de functie op dezelfde index uit de \texttt{functions} array uitgevoerd. De functie krijgt de input pointer mee als argument, waar 5 bij op wordt geteld. Dit is zodat de functie alleen het begin van de argumenten van het commando meekrijgt.

Wanneer de gebruiker echter maar 5 karakters heeft ingevoerd (Zoals \texttt{/dest}), lopen we tegen een probleem aan. De pointer die de functie meekrijgt ligt dan namelijk in de gebruikersinput buffer, \textit{na} de \textbackslash0 die wordt toegevoegd. Hierdoor weet de functie niet waar het einde van de gebruikersinput is, en kunnen er geheugenproblemen ontstaan. Dit is opgelost door een extra \textbackslash0 in de buffer te zetten wanneer de invoer maar 5 karakters lang is. Zo wordt aan de functie een pointer meegegeven naar een \textbackslash0, zodat deze kan weten dat de gebruiker geen verdere argumenten heeft meegegeven.

\begin{lstlisting}[caption={De code die het ingevoerde commando als functie uitvoert},captionpos=b,label={lst:interpretInputRun},style=c,xleftmargin=.\textwidth,xrightmargin=.\textwidth]
for(uint8_t i = 0; i < COMMANDS; i++) {
    if(strncmp(commands[i], input, 4) == 0) {
        // Prevent problems relating to inputs with no arguments.
        if(input[4] == '\0') input[5] = '\0';

        // Run the command, with the text after it as argument.
        functions[i](input + 5);
        return;
    }
}
printf("I don't know that command :(\n");
\end{lstlisting}



\paragraph*{Help}
Help print een gebruikersaanwijzing uit voor de gebruiker, als een lijst van alle commando's en hun functionaliteit.

\paragraph*{Chan}
Chan is een erg simpel commando. Het enige wat het doet is de frequentie veranderen waar de nrf chip op zendt en ontvangt. De werking van de functie is te zien in \autoref{lst:chan}.
\begin{lstlisting}[caption={De chan functie},captionpos=b,label={lst:chan},style=c,xleftmargin=.\textwidth,xrightmargin=.\textwidth]
void chan(char *arguments) {
    uint8_t channel = atoi(arguments);

    nrfStopListening();
    nrfSetChannel(channel);
    nrfStartListening();

    printf("\nSwitched to channel %d\n\n", channel);
}
\end{lstlisting}

De functie heeft geen afhandeling van een ongeldige input. Dit is omdat \texttt{atoi} altijd een 0 teruggeeft wanneer hij tegen een probleem aanloopt bij het parsen van een getal. Hierdoor wordt het kanaal gewoon naar 0 gezet wanneer de gebruiker een ongeldige invoer geeft. Hier wordt de gebruiker ook van op de hoogte gesteld, door de \texttt{printf} mededeling. Aangezien de kanaalfrequentie wordt berekent door middel van \autoref{eq:frequentie} is 0 gewoon een geldige kanaalfrequentie, en hoeft er geen actie ondernomen worden. Deze functie is in principe alleen bedoelt voor debugging doeleinden, aangezien de kanaalfrequentie al in de ISO is vastgesteld als 38 (2438 MHz).

\begin{equation} \label{eq:frequentie}
    f = 2400 + \textrm{kanaal [MHZ]})
\end{equation}

\paragraph*{Dest}
De dest functie werkt in principe hetzelfde als chan. Het enige verschil is dat dest het opgeslagen bestemmings-ID verandert in plaats van de frequentie. In \autoref{lst:dest} is te zien dat de dest functie wel ongeldige input afhandelt. Dit is omdat, waar 0 een geldige frequentie is bij /chan, het geen geldig ID is volgens de ISO. Het ID wordt uitgelezen met \texttt{strtol}, welke een hexadecimale string in een integer omzet. Dit is omdat ID's altijd als hexadecimaal getal genoteerd worden.

\begin{lstlisting}[caption={De dest functie},captionpos=b,label={lst:dest},style=c,xleftmargin=.\textwidth,xrightmargin=.\textwidth]
void dest(char *arguments) {
    if(arguments[0] == '\0') {
        printf("Usage:" COMMAND_COLOR " /dest" NO_COLOR " <ID>\n\n");
        return;
    }

    uint8_t newId = strtol(arguments, NULL, 16);
    if(newId == 0) printf(ERROR_COLOR "Invalid ID entered.\n\n" NO_COLOR);
    else {
        destinationId = newId;
        printf("New destination ID is " ID_COLOR "0x%02x" NO_COLOR "\n\n", newId);
    }
}
\end{lstlisting}


\paragraph*{Send}
De send functie stuurt encrypted alles wat er meegegeven wordt op via de \texttt{isoSendPacket} functie, zoals te zien in \autoref{lst:send}
\begin{lstlisting}[caption={De send functie},captionpos=b,label={lst:send},style=c,xleftmargin=.\textwidth,xrightmargin=.\textwidth]
void send(char *command) {
    // Set the rest of the inputbuffer to zero so that we don't accidentally 
    send previous messages with the message.
    uint8_t argLength = strlen(arguments);
    memset(arguments + argLength, 0, PAYLOAD_SIZE - argLength);

    uint8_t *data = keysEncrypt((uint8_t*)arguments, PAYLOAD_SIZE, key1Data, 
    strlen(key1Data), key2Data, strlen(key2Data));
    if(isoSendPacket(destinationId, (uint8_t*) data, PAYLOAD_SIZE)) 
        printf("Failed to send message\n");      
}
\end{lstlisting}


\paragraph*{List}
Het enige wat de list functie doet is de \texttt{printFriends} functie uitvoeren, te vinden in \autoref{ch:printFriends}.


\paragraph*{MyID}
MyID print het ID van de node naar de terminal. Dit is erg nuttig wanneer de geprinte header niet meer zichtbaar is door bijvoorbeeld een groot aantal ontvangen berichten.

\paragraph*{Keys}
Keys print key1 en key2 naar de terminal. Dit is erg nuttig wanneer het niet meer duidelijk was welke keys er werden gebruikt. Deze staan in Hexadecimaal inplaats van ASCII zodat de gebruiker nog een beetje moet ontcijferen wat de key ookalweer was.

\paragraph*{Key1}
Dit commando zet aan de array key1 de ASCII waarde van wat in de terminal is ingevoerd. 

\paragraph*{Key2}
Dit commando is hetzelfde als key1, maar doet dit dan voor key2.

\subsection{Terminal}
    \subsubsection{Intrepeteer input}
    \subsubsection{Command callback}
    \subsubsection{Terminal print}
    \subsubsection{Terminal print hex}
    \subsubsection{Terminal print strex}


    