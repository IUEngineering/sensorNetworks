Vanuit verschillende onderzoeken is er gekeken naar invloeden van buiten om de productiviteit van een persoon te beinvloeden. \cite{productiviteit} 
Hieruit is de volgende lijst en dus ook grootheden gekomen:

\begin{table}[ht]
    \begin{tabular}{p{3.3cm}||p{3cm}|p{3cm}|p{3cm}|p{3cm}}
         & Eenheid & Meetbereik & Nauwkeurigheid & meetfrequentie\\
         \hline
        Temperatuur &  $^{\circ}$C  & 10 tot 29 $^{\circ}$C & 90 \% & elk half uur\\
        Luchtvochtigheid & \% & 0 tot 100\%& 80 \% & elk half uur\\
        Luchtkwaliteit & Groen, Oranje en rood led & zie kleur tabel*& 80 \% & elke 10 minuten\\
        Geluid in de ruimte & dB & 0 - 90dB & 95 \% & elke 5 seconde**\\
        Licht & Lumen & 0 - 1600Lm & 90\% & elke 10 seconde
    \end{tabular}
    \caption{Grootheden}
    \label{tab:1}
\end{table}


 \vspace{1cm}
Wanneer 1 van deze grootheden bij een bepaalde waarde komt dat het niet meer comfortabel is, zie tabel \ref{tab:3}, dan zal er eerst een waarschuwing naar voren komen. Deze waarschuwing kan als knipperend lichtje of een kleine piep zijn. Wanneer meerdere grootheden tegelijketijd in de oncomfortabele zone zitten dan zal er een heftigere waarschuwing zijn. 

\vspace{1.5cm}
* Kleur tabel, kleuren inspiratie vanuit het KNMI \cite{Gezonde} \\

\begin{table}[ht]
    \begin{tabular}{c|c|c}
       Kleur & Eenheid & Meetbereik \\
       \hline
       Groen  &  microgram per cubic meter & $ < $ 16\\
       Oranje  &  microgram per cubic meter & 16 $->$ 25\\
       Rood & microgram per cubic meter & $>$ 25
    \end{tabular}
    \caption{kleur tabel}
    \label{tab:2}
\end{table}

\vspace{0.2cm}

** Hier moet een filter bij komen omdat er in een split seconde veel geluid kan zijn voor een seconde en dit niet meteen de alarmbellen moet afslaan. Een voorbeeld zou zijn om het gemiddelde van elke minuut te pakken.

\begin{table}[ht]
    \begin{tabular}{p{3cm}||p{3cm}|p{3cm}}
       Grootheid  & ondergrens & bovengrens \\
       \hline
       Temperatuur  & $<$ 15& $>$ 22 \cite{Temperatuur}\\
       Luchtvochtigheid & $<$ 40\% & $>$60\% \cite{Luchtvochtigheid}\\
       Luchtkwaliteit & n.v.t. & Rood \cite{Gezonde}\\
       Geluid in de ruimte & n.v.t. & $>$ 60dB \cite{Geluidsoverlast}\\
       Licht & n.v.t & $>$ 1500Lm \cite{Lumen}\\
    \end{tabular}
    \caption{Grenzen per grootheid}
    \label{tab:3}
\end{table}